\section{Evaluation}
\label{sec:evaluation}

We evaluate AC-BlockDFL through systematic experiments designed to validate its defense effectiveness against Progressive Committee Capture Attacks (PCCA). Rather than treating model accuracy as the primary metric, our evaluation focuses on whether the economic security mechanism effectively deters rational adversaries and maintains long-term governance stability. This perspective shift reflects our core design philosophy: when the defense objective transitions from ``preventing attacks'' to ``ensuring attacks are unprofitable,'' the evaluation metrics should correspondingly shift from model quality to the adversary's economic decision space.

Our experiments adopt a worst-case analysis methodology, assuming adversaries attack whenever possible regardless of economic rationality. This design enables a critical inference: if the mechanism ensures every attack is detected and penalized under worst-case conditions, rational adversaries will preemptively conclude that expected returns are negative and abstain from attacking, allowing the system to naturally converge toward stable equilibrium.

\subsection{Experimental Setup}
\label{sec:eval_setup}

We use the MNIST dataset with a standard CNN (two convolutional layers, two fully connected layers) as our federated learning testbed. Training data is distributed IID across clients---a deliberate choice since our defense operates at the consensus layer rather than the data layer. Committee composition and voting outcomes determine attack success, which are logically independent of underlying data distribution characteristics.

\begin{table}[t]
    \centering
    \caption{Experimental Parameters}
    \label{tab:exp_params}
    \small
    \begin{tabular}{ll}
        \toprule
        \textbf{Parameter} & \textbf{Value} \\
        \midrule
        Training rounds & $R = 300$ (baseline) / $R = 2000$ (long-term) \\
        Validator pool size & $N = 100$ \\
        Committee size & $C = 7$ \\
        Malicious nodes & $M = 30$ (initial stake ratio $30\%$) \\
        Per-round rewards & Validator: 1.0, Aggregator: 1.0, Provider: 0.05 \\
        Slashing rule & Full stake confiscation upon successful challenge \\
        \bottomrule
    \end{tabular}
\end{table}

Table~\ref{tab:exp_params} summarizes the experimental configuration. The $30\%$ initial malicious stake ratio represents a severe threat scenario approaching the theoretical tolerance limit of most Byzantine fault-tolerant systems. Under hypergeometric distribution analysis (Section~\ref{sec:committee-size-security}), malicious nodes have approximately $2.4\%$ probability of capturing $\geq 5$ of 7 committee seats in any single round. While seemingly modest, this probability accumulates over hundreds to thousands of training rounds, providing ample attack opportunities for rigorous defense validation.

\subsection{Committee Security Analysis} \label{sec:committee-size-security}
The probability of $k$ malicious nodes being selected in a committee of size $C$ from a pool $N$ with $M$ malicious nodes follows the hypergeometric distribution:
\begin{equation}
    P(X=k) = \frac{\binom{M}{k} \binom{N-M}{C-k}}{\binom{N}{C}}
\end{equation}
With $N=100, M=30, C=7$, the probability of an adversarial takeover ($k \geq 5$) is $\sim 2.41\%$.

\subsection{Long-term Governance Equilibrium}
\label{sec:macro_equilibrium}

The critical question our evaluation addresses is whether short-term penalty effectiveness translates into long-term governance stability where attacks naturally cease. The 2000-round simulation provides definitive evidence.

\begin{figure}[t]
    \centering
    \includegraphics[width=0.85\columnwidth]{figures/experiments/mnist_results_stack_comparison_2000_round.png}
    \caption{Stake dynamics over 2000 rounds. BlockDFL exhibits persistent governance imbalance with malicious stake ratio stabilizing at 1.3$\times$, while AC-BlockDFL achieves progressive purification through five slashing events, reducing malicious stake to 0.37$\times$ of honest nodes.}
    \label{fig:stake_evolution_2000}
\end{figure}

Figure~\ref{fig:stake_evolution_2000} reveals fundamentally divergent governance trajectories. In BlockDFL, the malicious stake ratio stabilizes around 1.3 after initial fluctuations and persists throughout the experiment. This seemingly modest advantage masks a profound governance crisis: the 1.3$\times$ ratio translates to significantly elevated committee election probabilities, sustaining continuous attack capability across 2000 rounds. Without accountability mechanisms, adversaries reinforce their stake advantage through each successful capture, confirming the positive feedback loop predicted in Section~\ref{sec:pcca}.

AC-BlockDFL exhibits a starkly different pattern. The malicious stake ratio undergoes five distinct step-wise decreases at rounds 15, 136, 695, 815, and 1332, declining from the initial 1.0 to a final 0.37. This terminal value indicates that malicious nodes retain barely one-third the average stake of honest participants---a $1.3/0.37 \approx 3.5\times$ difference from BlockDFL representing fundamental governance reversal rather than incremental improvement.

\begin{table}[t]
    \centering
    \caption{Attack Event Statistics (2000-round Simulation)}
    \label{tab:attack_events_stats}
    \small
    \begin{tabular}{lcc}
        \toprule
        \textbf{Metric} & \textbf{BlockDFL} & \textbf{AC-BlockDFL} \\
        \midrule
        Total attacks & 107 & 5 \\
        Strategic starvation & 18 & 2 \\
        Full-stack poisoning & 89 & 3 \\
        Detected \& slashed & 0 & 5 (100\%) \\
        Final stake ratio & 1.30 & 0.37 \\
        \bottomrule
    \end{tabular}
\end{table}

Table~\ref{tab:attack_events_stats} provides direct causal evidence for the stake trajectories. BlockDFL records 107 committee capture events (averaging one per 19 rounds), none receiving economic sanction. AC-BlockDFL records only 5 attacks, all successfully detected and penalized---a $>$20$\times$ reduction stemming from two reinforcing mechanisms: slashing directly depletes the malicious stake base, reducing subsequent committee election probability; simultaneously, diminished stake ratios raise the difficulty of achieving $\geq$5/7 committee control even when selected.

The increasing intervals between slashing events provide key evidence of convergence toward equilibrium. Specifically: 121 rounds between events 1--2, 559 rounds between 2--3, 120 rounds between 3--4, 517 rounds between 4--5, and 668 rounds of silence following the fifth event through experiment termination. This pattern is not statistical noise but a mathematical consequence of stake depletion: as malicious stake fraction decreases from $30\%$ toward $20\%$, the single-round probability of achieving committee control drops from $\sim$2.4\% to $<$0.5\%, directly manifesting as attack window rarefaction. The 668-round silent period following the final slashing confirms the system has converged to a state where attacks become structurally improbable.

\subsection{Immediate Mechanism Response}
\label{sec:micro_mechanism}

The 300-round baseline experiment provides a controlled window for examining the immediate impact of individual slashing events on governance structure.

\begin{figure}[t]
    \centering
    \includegraphics[width=0.85\columnwidth]{figures/experiments/mnist_results_stack_comparison.png}
    \caption{Stake evolution comparison (300-round baseline). AC-BlockDFL demonstrates immediate stake ratio drops upon each slashing event, while BlockDFL shows continuous malicious stake accumulation.}
    \label{fig:stake_evolution_300}
\end{figure}

Figure~\ref{fig:stake_evolution_300} shows the early-stage stake trajectories. In BlockDFL, 10 committee capture events occur over 300 rounds (4 strategic starvation, 6 full-stack poisoning), all unpunished, enabling the malicious stake ratio to climb steadily from 1.0 toward 1.15. AC-BlockDFL records only 2 attacks at rounds 90 and 229, both detected and slashed with 100\% accuracy.

The first slashing event illustrates the mechanism's precision. By round 90, malicious nodes had accumulated a 1.25 stake ratio through honest participation, translating to elevated committee selection probability. When 5 malicious nodes achieved committee control and executed full-stack poisoning, a challenger detected the deviation by locally re-executing Krum aggregation and submitted a challenge transaction. Upon arbitration confirmation, the smart contract automatically confiscated the full stakes of all 5 colluding members. The economic impact was immediate and severe: the malicious stake ratio plummeted from 1.25 to 0.62---a single event reversing the adversary's 25\% lead into a 38\% deficit.

This magnitude of impact warrants careful interpretation. The slashed nodes lost not merely the current round's potential gains (bounded by $\sim$7.0 reward units) but their entire accumulated stake from 89 rounds of honest participation. More critically, stake-zeroed nodes are effectively excluded from future high-reward role elections, constituting ``permanent governance exclusion'' that degrades long-term attack capability beyond the immediate economic penalty.

The second slashing at round 229 reduced the stake ratio from 0.70 to 0.52. The 139-round interval between attacks (versus BlockDFL's average of 30 rounds) directly reflects the first slashing's suppressive effect on attack opportunity windows.

\subsection{Service Quality Under Security Guarantees}
\label{sec:service_quality}

A critical concern is whether security guarantees impose unacceptable performance costs. We evaluate both system availability and model convergence quality.

\paragraph{System Availability.} We define minimum unavailability rate as the fraction of rounds where model performance is significantly degraded due to full-stack poisoning attacks. Each attack requires approximately 5--25 rounds for federated learning's self-healing mechanism to restore accuracy. Using the conservative 5-round estimate, BlockDFL's 89 full-stack attacks yield a minimum unavailability rate of $89 \times 5 / 2000 = 22.3\%$. AC-BlockDFL achieves $3 \times 5 / 2000 = 0.75\%$---a $>$96\% improvement attributable entirely to attack frequency suppression rather than enhanced per-attack resilience.

\begin{figure}[t]
    \centering
    \includegraphics[width=0.85\columnwidth]{figures/experiments/mnist_results_convergence.png}
    \caption{Model accuracy convergence comparison. AC-BlockDFL exhibits smoother training dynamics with fewer disruption-recovery cycles.}
    \label{fig:convergence_curve}
\end{figure}

\paragraph{Model Convergence and Training Stability.} 
Figure~\ref{fig:convergence_curve} compares accuracy trajectories over the 300-round baseline. BlockDFL's accuracy curve exhibits pronounced sawtooth patterns, with each severe drop precisely corresponding to a full-stack poisoning attack. Notably, the earliest threats in BlockDFL did not manifest as accuracy drops. The initial attack at round 69 was a stealthy ``strategic starvation'' maneuver. By rejecting honest proposals and approving suboptimal ones that favored malicious providers, the adversary manipulated the economic flow to accelerate stake accumulation. From an accuracy monitoring perspective, strategic starvation leaves almost no anomalies, confirming our argument (Section~\ref{sec:pcca}) that relying solely on model quality metrics creates a fundamental security blind spot. As malicious nodes built their stake advantage through starvation, attack frequency accelerated (averaging one per 19 rounds), eventually escalating to full-stack poisoning. Although federated learning's resilience allows gradual recovery, this continuous ``deviation and correction'' cycle severely degrades computational efficiency.

Conversely, AC-BlockDFL's accuracy curve demonstrates fundamentally different dynamic stability. During the 300-round period, it suffered only 2 full-stack attacks. The extreme case at round 90 provides an excellent benchmark: the adversary executed a label-flipping attack that crashed accuracy from normal levels down to 9.5\% (near random-guess baseline for MNIST's 10-class task). However, the system showcased remarkable self-healing capability, recovering to pre-attack levels within $\approx 20$ rounds. More importantly, as training progressed and the slashing mechanism purged malicious stake, the system's resilience improved. In late-stage training, the robust parameter structure established during uninterrupted honest epochs reduced the recovery time for equivalent disturbances from 20 rounds down to just $\approx 5$ rounds. 

Both architectures reach similar final accuracies (98.26\% vs. 98.63\%), confirming that the ``no-rollback'' design philosophy (Section~\ref{sec:no_rollback}) is practically sound: federated learning's iterative nature inherently digests occasional deviations, obviating the immense coordination overhead of state rollbacks. Yet, AC-BlockDFL's near-interference-free environment ensures computing resources are efficiently translated into optimization gains rather than wasted on repairing attack damage—a critical advantage for resource-constrained edge deployments.

\paragraph{Communication Efficiency.} \label{sec:efficiency_analysis}
As analyzed in Section~\ref{sec:efficiency_analysis} and summarized in Table~\ref{tab:efficiency_comparison}, AC-BlockDFL achieves $O(C^2)$ communication complexity under equilibrium conditions where the challenge trigger probability $p \to 0$. Compared to approaches requiring equivalent security guarantees through full replication, this represents approximately 40\% reduction in per-round communication overhead while maintaining the same Byzantine tolerance threshold.

\begin{table}[t]
    \centering
    \caption{Communication Complexity Comparison}
    \label{tab:efficiency_comparison}
    \small
    \begin{tabular}{lcc}
        \toprule
        \textbf{Scheme} & \textbf{Complexity} & \textbf{Overhead (MB/round)} \\
        \midrule
        Full BFT & $O(N^2)$ & 25.4 \\
        BlockDFL & $O(C^2)$ & 4.2 \\
        AC-BlockDFL & $O(p N^2 + C^2)$ & 4.3 \\
        \bottomrule
    \end{tabular}
\end{table}

\subsection{Summary}

Our evaluation validates AC-BlockDFL's defense effectiveness through three complementary lenses. At the micro level, each malicious committee decision triggers immediate detection and slashing with 100\% accuracy. At the macro level, five slashing events progressively reduce the malicious stake ratio from 1.0 to 0.37, with increasing inter-event intervals and a terminal 668-round silent period confirming convergence to attack-free equilibrium. Service quality analysis demonstrates that these security guarantees impose minimal performance cost: unavailability rate drops from 22.3\% to 0.75\%, while model convergence remains uncompromised. These results complete the inference chain: worst-case testing proves all attacks are detected; rational adversaries therefore anticipate penalties and abstain; the system operates at designed efficiency under the resulting equilibrium.