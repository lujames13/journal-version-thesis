\section{Related Work}
\label{sec:related_work}

The convergence of federated learning and blockchain technology has precipitated diverse architectural innovations to address decentralized coordination, privacy, and security. Early systems like DeepChain~\cite{weng2021deepchain} and Biscotti~\cite{shayan2021biscotti} focused on preserving privacy during aggregation using cryptographic commitments and differential privacy. However, scaling BFT consensus to accommodating hundreds of FL clients remained a persistent challenge due to its $O(N^2)$ communication overhead.

\subsection{Evolution of Committee Architectures}
To mitigate consensus bottlenecks, recent literature has pivoted toward committee-based designs inspired by Algorand's sortition~\cite{gilad2017algorand}. By randomly selecting a constant-size committee to perform consensus, systems effectively decouple performance from total network size. FLCoin~\cite{ren2024scalable} utilized a sliding-window mechanism based on contribution history to form dynamic committees, achieving up to 90\% reduction in communication overhead. Similarly, BFLC~\cite{li2021blockchain} adopted a reputation-based election scheme, prioritizing nodes with high historical quality scores.

While optimization successes are evident, these election mechanisms inherently couple system security to committee composition. If an adversary captures a supermajority (e.g., $>2/3$) of the committee seats, traditional data-layer defenses like Krum~\cite{blanchard2017machine} or Trimmed Mean~\cite{yin2018byzantine} are entirely bypassed because the compromised committee itself executes these algorithms.

\subsection{Limitations of Existing Verification Methods}
Addressing Byzantine behavior in decentralized aggregation currently relies on three primary verification paradigms:

\paragraph{Cryptographic Verification (zkML).}
Zero-Knowledge Machine Learning (zkML) provides the strongest security guarantees by compiling learning computations into arithmetic circuits, allowing verification without re-execution~\cite{chen2024zkml}, \cite{zhu2024risefl}. However, zkML faces prohibitive computational bottlenecks. Compiling simple architectures like ResNet-18 generates millions of polynomial constraints, and generating proofs takes minutes. More critically, zkML currently cannot support complex, non-linear Byzantine-robust aggregation algorithms like Krum, which require $O(N^2 \cdot d)$ pairwise distance calculations that cause circuit explosions.

\paragraph{Optimistic Execution (opML).}
Optimistic Machine Learning (opML) defaults to accepting computations but allows a challenge window during which "AnyTrust" challengers can submit fraud proofs via interactive bisection protocols~\cite{conway2024opml}, \cite{ora2024opml}. While efficient, mainstream opML architectures natively resolving disputes on-chain require challenge periods extending up to a week (e.g., Optimism~\cite{optimism2024rollup}). These latencies fundamentally conflict with the high-frequency iterative nature of federated learning, rendering opML broadly inapplicable for per-round FL aggregation.

\paragraph{Committee Consensus and Static Blindspots.}
Committee-based verification remains the most pragmatic solution but is underpinned by static probabilistic assumptions: current security analyses calculate committee capture probability by assuming a fixed adversarial population distribution~\cite{ren2024scalable}. This static view completely overlooks the behavior of \emph{rational, strategic adversaries}. As indicated by our analysis of the stake-election-reward cycle, systems like BlockDFL~\cite{qin2024blockdfl} and FedBlock~\cite{nguyen2024fedblock} contain positive feedback loops. Strategic attackers can exploit this by behaving honestly ("lurking") to accumulate stake and reputation until they acquire sufficient influence to capture the committee ("starving" honest participants). 

\subsection{Our Contribution}
Our work directly addresses the systemic blindspot in static committee security. We define and analyze the Progressive Committee Capture Attack (PCCA), demonstrating how rational adversaries bypass conventional defenses. In contrast to existing static committee systems, AC-BlockDFL breaks the stake feedback loop through optimistic execution coupled with an asynchronous, stake-slashing auditing layer, securing the system economically while preserving the efficiency benefits of committee architectures.
