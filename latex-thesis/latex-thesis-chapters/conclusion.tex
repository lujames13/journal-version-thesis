\section{Conclusion}
\label{sec:conclusion}

This paper identifies and formalizes the Progressive Committee Capture Attack (PCCA), demonstrating how rational adversaries can systematically compromise committee-based blockchain federated learning systems through strategic stake accumulation. Our long-horizon simulations confirm that conventional committee architectures exhibit stake ossification and governance capture under sustained attack.

To address this threat, we propose AC-BlockDFL, an audit-driven committee architecture that decouples security guarantees from committee size. The key insight underlying our design is a paradigm shift from \emph{threshold security}---which seeks to minimize the probability of committee compromise---to \emph{economic security}---which ensures that even successful compromise yields negative expected utility for rational adversaries. Through asynchronous auditing and the internal slashing protocol, AC-BlockDFL achieves progressive purification of malicious participants while maintaining the efficiency benefits of small committees.

Our experimental results validate three principal contributions: (1) formal threat modeling of PCCA with empirical verification of its feasibility; (2) demonstration that slashing mechanisms effectively break the positive feedback loop of malicious stake accumulation, internalizing the externalities of adversarial behavior; and (3) evidence that shifting from preventive to reactive security breaks the tight coupling between security guarantees and communication overhead, enabling practical deployment in resource-constrained edge computing scenarios.