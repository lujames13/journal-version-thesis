\section{Background}
\label{sec:background}

This section establishes the theoretical foundations and technical background necessary for understanding the security challenges in committee-based blockchain federated learning. We first discuss the fundamental trust dilemma in federated learning, then introduce the principles of Byzantine fault tolerance, and finally establish the baseline system model for committee-based architectures.

\subsection{Federated Learning and the Trust Dilemma}
\label{sec:fl-trust}

Federated learning (FL) represents a paradigm shift in distributed machine learning, encapsulating the principle of ``bringing the model to the data'' rather than aggregating data centrally~\cite{mcmahan2017communication}. While FL significantly enhances data privacy by locally constraining raw data, its standard architecture relies on a fundamental assumption: participants must trust a central aggregation server to honestly execute aggregation and uniformly distribute results. 

In the absence of verifiable consistency, the central server constitutes a single point of failure and a primary vulnerability. A malicious or compromised server could perform selective aggregation, intentionally excluding specific updates, or directly tamper with the global model to inject backdoors~\cite{bagdasaryan2020how}. Furthermore, while FL avoids direct data transmission, a malicious aggregator can still infer sensitive information from client updates~\cite{zhu2019deep}. This trust dilemma severely restricts the deployment of FL in high-value, cross-organizational scenarios where participating entities may be independent or competitive, necessitating a decentralized trust infrastructure.

\subsection{Byzantine Fault Tolerance Fundamentals}
\label{sec:bft-fundamentals}

Blockchain technology, characterized by immutability, transparency, and decentralization, serves as an ideal infrastructure to resolve the FL trust dilemma. However, the security of blockchain fundamentally relies on consensus protocols designed to tolerate malicious behavior, rooted in the Byzantine Generals Problem~\cite{lamport1982byzantine}.

The mathematical constraint of Byzantine Fault Tolerance (BFT) dictates that a system of $N$ nodes can tolerate at most $f$ malicious nodes, requiring $N \geq 3f + 1$. This one-third threshold originates from the \emph{quorum intersection principle}: to ensure sufficient honest endorsements, any decision needs $2f+1$ confirmations. The intersection of any two $2f+1$ sets guarantees the inclusion of at least $f+1$ nodes, meaning at least one honest node witnesses both decisions, preventing contradictory states.

Practical Byzantine Fault Tolerance (PBFT)~\cite{castro1999practical} reduces the communication complexity of BFT consensus to $O(N^2)$ through a three-phase commit protocol (Pre-prepare, Prepare, Commit). While efficient for small networks, the quadratic communication cost becomes a severe bottleneck for large-scale FL systems requiring frequent iterations involving hundreds of participants.

\subsection{Committee-based BCFL Architecture}
\label{sec:committee-bcfl}

To reconcile the efficiency demands of federated learning with the security requirements of blockchain, the \emph{committee-based architecture} has emerged as the prevailing design paradigm. By delegating consensus responsibilities to a smaller, representative subset of nodes (the committee), these systems reduce communication complexity from $O(N^2)$ to $O(C^2 + N)$ where the committee size $C \ll N$~\cite{ren2024scalable}.

\paragraph{Baseline System Model: BlockDFL.} \label{sec:blockdfl-baseline}
We adopt BlockDFL~\cite{qin2024blockdfl} as our baseline system model, representing the state-of-the-art in peer-to-peer BCFL. BlockDFL implements role separation, partitioning participants into three distinct roles per training round:
\begin{enumerate}
    \item \textbf{Update Providers}: Execute local model training on private data and submit bounded updates.
    \item \textbf{Aggregators}: Collect updates, perform filtering, and compute the aggregated global proposal.
    \item \textbf{Validators}: Form the committee to evaluate competing proposals via Krum scoring~\cite{blanchard2017machine} and execute PBFT consensus to select the final model.
\end{enumerate}

\paragraph{The Stake-Election-Reward Cycle.}
Role assignment in BlockDFL relies on \emph{stake-weighted deterministic random selection}, using the previous block's hash as an unpredictable entropy source mapping onto a stake-weighted hash ring. This creates a critical economic incentive structure designed to solve the free-rider problem:
\begin{itemize}
    \item \textbf{Stake}: Determines election probability; participants with higher stake bounds are proportionally more likely to be selected as Aggregators or Validators.
    \item \textbf{Reward}: Distributed exclusively to contributors of the accepted proposal (the winning Aggregator, included Update Providers, and Validators who voted for it).
\end{itemize}
This mechanism instantiates a \emph{positive feedback loop}: receiving rewards increases absolute stake, which enhances future election probability and aggregation weight, thereby amplifying the likelihood of subsequent rewards. While intended to cultivate long-term honest contributions, this very dynamic introduces vulnerabilities to strategic persistence.
