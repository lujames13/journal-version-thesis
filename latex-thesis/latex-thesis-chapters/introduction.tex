\section{Introduction}
\label{sec:introduction}

Blockchain-based Federated Learning (BCFL) has emerged as a promising paradigm for collaborative machine learning in environments where mutual trust among participants cannot be assumed. Real-world deployments in Low Earth Orbit (LEO) satellite networks~\cite{pokhrel2021blockchain, wu2024sharded, elmahallawy2025decentralized}, vehicular networks (V2X)~\cite{liu2021blockchain, pokhrel2020autonomous}, and Industrial IoT~\cite{lu2020blockchain, qu2020decentralized} demonstrate compelling use cases where decentralized coordination is essential. In LEO constellations, for instance, ground station contact windows last merely five minutes with downlink bandwidth limited to approximately 8 Mbps~\cite{wu2024sharded}, rendering centralized aggregation architectures impractical. BCFL addresses these constraints by establishing decentralized trust infrastructure across heterogeneous satellite operators, reducing model convergence time by up to thirty hours~\cite{elmahallawy2025decentralized}.

However, BCFL systems face a fundamental scalability bottleneck when approaching large-scale deployment. The predominant use of Practical Byzantine Fault Tolerance (PBFT)~\cite{castro1999practical} and its variants introduces $O(N^2)$ message complexity, causing consensus latency to dominate training time as participant counts grow. Empirical measurements from FLCoin~\cite{ren2024scalable} reveal that at 100 nodes, single-round consensus generates over 20,000 message exchanges with latency exceeding 25 seconds---comparable to or exceeding the model training duration itself. Storage requirements compound this challenge: Bitcoin full nodes require approximately 200 GB while Ethereum exceeds 465 GB, fundamentally incompatible with edge devices possessing only KB-to-MB scale memory~\cite{fedchain2024}.

\paragraph{The Committee Mechanism.} To address these scalability constraints, recent work has converged on \emph{committee-based} architectures that delegate verification responsibility to a smaller subset of validators. Selection mechanisms include hash-ring sampling~\cite{qin2024blockdfl}, stake-weighted election~\cite{li2021blockchain, ren2024scalable}, and Verifiable Random Function (VRF) based sortition~\cite{shayan2021biscotti, weng2021deepchain, gilad2017algorand}. These approaches yield substantial efficiency gains: FLCoin~\cite{ren2024scalable} reports 90\% communication overhead reduction and 5.7$\times$ training speedup through sliding-window election, while BFLC~\cite{li2021blockchain} achieves sub-three-second consensus latency. Effectively, committee mechanisms reduce communication complexity from $O(N^2)$ to $O(C^2)$ or even $O(C)$, where $C \ll N$ is the committee size.

\paragraph{The Blind Spot.} Despite these advances, existing BCFL literature harbors a critical yet overlooked vulnerability: the implicit assumption that committee members remain honest or that the proportion of malicious nodes stays static throughout system operation. Current defenses focus predominantly on data-plane attacks---Byzantine-robust aggregation rules such as Krum, Trimmed Mean, and Median~\cite{blanchard2017machine, yin2018byzantine} assume an honest majority among aggregating nodes. However, these mechanisms provide no protection when the committee itself becomes compromised. As FedBlock~\cite{nguyen2024fedblock} observes, when any participant may become a validator, systems cannot rely solely on honest majority assumptions but must actively detect and isolate malicious verifiers---a capability conspicuously absent from current BCFL architectures.

\paragraph{The PCCA Threat.} We identify and formalize a novel attack vector: the \emph{Progressive Committee Capture Attack} (PCCA). Unlike direct Byzantine attacks, PCCA adversaries employ \emph{strategic starvation}---upon gaining committee control, attackers prioritize processing their own model updates while systematically denying service to honest participants. This manipulation of the reward distribution mechanism enables attackers to ``legitimately'' accumulate stake over successive rounds, progressively cementing their dominance until decentralized governance collapses entirely. Crucially, once the honest majority assumption fails in any given round, existing systems lack mechanisms to identify or penalize malicious actors, allowing attackers to maintain their advantage indefinitely.

\paragraph{Contributions.} This paper presents \emph{Audit-driven Committee BlockDFL} (AC-BlockDFL), a defense framework that decouples system security from collective committee honesty through optimistic execution with asynchronous auditing. Our contributions are threefold:

\begin{enumerate}
    \item \textbf{Attack Formalization.} We provide the first formal definition of the Progressive Committee Capture Attack (PCCA) and a rational adversary model that captures strategic, incentive-driven behavior. Through systematic simulation, we quantify PCCA's destructive impact on long-term incentive compatibility.
    
    \item \textbf{AC-BlockDFL Architecture.} We propose a novel defense architecture combining optimistic execution with asynchronous auditing. A distributed challenger network performs post-hoc verification of committee decisions, enabling detection and penalization of fraudulent aggregation results even when the committee is fully compromised.
    
    \item \textbf{Game-Theoretic Guarantees.} We design an internal slashing protocol grounded in game-theoretic analysis, ensuring that auditing costs remain strictly below potential gains from malicious behavior. We prove that honest participation constitutes the unique Nash equilibrium under repeated play~\cite{chiu2018incentive}. Extensive simulations over 2,000 rounds demonstrate that AC-BlockDFL maintains model accuracy above 98.6\% under 30\% adversarial collusion, reduces communication overhead by 44.4\% at equivalent security levels, and suppresses minimum unavailability rate from 20\% to below 5\%.
\end{enumerate}